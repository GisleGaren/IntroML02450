\documentclass[twoside, 11pt]{article}

\usepackage[preprint]{jmlr2e}
\usepackage[a3paper]{geometry}
\setcounter{page}{9}


\begin{document}

\appendix
\section{Description of dataset}

\begin{table}[h!]
	\centering
	\begin{tabular}{|l|l|l|p{8cm}|}
		\hline
		\textbf{Attribute} & \textbf{Discrete/ Continuous} & \textbf{Type} & \textbf{Explanation} \\ \hline
		Survived & Discrete & Nominal & This is a binary attribute (0 = did not survive, 1 = survived) with no inherent order between the categories. It simply represents a classification into two distinct groups. \\ \hline
		Pclass (Passenger Class) & Discrete & Ordinal & Pclass indicates the socio-economic class of the passenger (1 = first class, 2 = second class, 3 = third class). Although it is a discrete variable, it is ordinal because there is a clear hierarchy (first class is higher in rank than second or third). \\ \hline
		Name & Discrete & Nominal & The Name attribute is a string (text) variable that uniquely identifies passengers. Since there is no inherent order or numerical relationship between names, it is nominal. \\ \hline
		Sex & Discrete & Nominal & The Sex attribute represents the gender of the passenger (male/female). This is a nominal variable as the categories are distinct and there is no order. \\ \hline
		Age & Continuous & Ratio & Age is a continuous variable that represents the passenger’s age. Since age has a meaningful zero point (birth) and the differences between values are consistent, it is measured on a ratio scale. \\ \hline
		SibSp (Number of Siblings/Spouses Aboard) & Discrete & Ratio & SibSp is a count of how many siblings or spouses a passenger had aboard. It is discrete because it represents a count, and ratio because it has a meaningful zero and equal intervals. \\ \hline
		Parch (Number of Parents/Children Aboard) & Discrete & Ratio & Parch is a count of how many parents or children a passenger had aboard. Like SibSp, it is a discrete variable on a ratio scale with a meaningful zero. \\ \hline
		Ticket & Discrete & Nominal & The Ticket attribute is an identifier for the passenger's ticket. It is nominal because it consists of text or numerical codes that do not have any intrinsic order. \\ \hline
		Fare & Continuous & Ratio & Fare represents the amount of money the passenger paid for the ticket. It is a continuous variable with a meaningful zero (no fare), and the differences between values are meaningful, making it a ratio variable. \\ \hline
		Cabin & Discrete & Nominal & The Cabin attribute is a string that represents the cabin number assigned to the passenger. It is nominal because the cabin numbers are simply labels with no inherent numerical or ordered relationship. \\ \hline
		Embarked (Port of Embarkation) & Discrete & Nominal & This attribute indicates the port where the passenger boarded the Titanic (C = Cherbourg, Q = Queenstown, S = Southampton). Since there is no natural ordering between these ports, it is a nominal variable. \\ \hline
	\end{tabular}
	\caption{Summary of Titanic Dataset Attributes by Type and Measurement Scale}
	\label{table:description}
\end{table}

\end{document}
