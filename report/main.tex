\documentclass[twoside,11pt]{article}

\usepackage[preprint]{jmlr2e}
\usepackage{amsmath}
\allowdisplaybreaks
\usepackage{enumitem}
\usepackage{lastpage}

\makeatletter

\let\@oldsection\section
\renewcommand\section[1]{\@oldsection*{#1}}

\let\@oldsubsection\subsection
\renewcommand\subsection[1]{\@oldsubsection*{\textit{#1}}}

\makeatother

\title{02450 Report 1}


\begin{document}

\maketitle\vspace{-3em}

\section{Table of authors and their contributions}

\begin{center}
	\begin{tabular}{| r | l | l | l |}
		\hline
				Name & Gisle Joe Garen & Ignacio Ripoll González & Sebastian Svelmøe Timm\\
		Student ID   & s242712 & s242875 & s243935\\
		\hline
		Introduction & 10\% & 80\% & 10\%\\
			  Task 1 & 10\% & 80\% & 10\%\\
			  Task 2 & 45\% & 45\% & 10\%\\
			  Task 3 & 80\% & 10\% & 10\%\\
			  Task 4 & 10\% & 10\% & 80\%\\
			Problems & 10\% & 10\% & 80\%\\
		\hline
	\end{tabular}
\end{center}

\section{Introduction}

TBD

\section{Task 1}

\subsection{one, two, three, this is a test}

The Titanic dataset not only provides a valuable resource for statistical analysis but also offers a glimpse into one of the most infamous maritime disasters in history. The RMS Titanic set sail from Southampton, England, on April 10, 1912, bound for New York City. On board were over 2,200 passengers and crew. Tragically, on the night of April 14, 1912, the Titanic struck an iceberg and sank within a few hours. Of the more than 2,200 people on board, over 1,500 lost their lives.

The ship sank due to a combination of design flaws, including insufficient watertight compartments, and the collision with the iceberg, which breached the hull. A critical factor in the high death toll was the lack of lifeboats. The Titanic carried lifeboats for only about half of those aboard, meaning that when the ship began to sink, not everyone had a chance to escape.

The overall problem of interest in studying the Titanic dataset is to identify which demographic, socio-economic, and other relevant characteristics influenced survival rates during the disaster. The dataset includes information on passenger class, gender, age, fare, family size, and more, allowing us to analyze patterns and key factors affecting survival probabilities. It is particularly compelling because the lifeboat shortage necessitated prioritizing passengers, with those in higher social classes, such as first-class passengers, being given precedence during evacuation. By examining these factors, we can gain insights into how class, age, gender, and other variables impacted survival, revealing broader socio-economic dynamics and human behavior in life-and-death situations.



The Titanic dataset was obtained from Kaggle, a popular platform for data science competitions and datasets. You can access the data via the following link: Kaggle Titanic Dataset. The data is provided in two separate datasets: one for training and one for testing. The training dataset includes the information used to build predictive models, while the testing dataset is used to evaluate the performance of these models.



\section{Task 2}

TBD

\section{Task 3}

TBD

\section{Task 4}

TBD

\section{Exam Problems}

\subsection{Question 1}

The only correct statement is C. The justification is:

\begin{itemize}
\item
  The variable \(x_1\) represents the time of day in blocks of 30
  minutes that partition a day into a finite number of intervals. The
  variable is therefore discrete. Since the intervals can be ordered,
  the variable is ordinal.
\item
  The attribute \(x_6\) is a ratio variable. The number of broken
  traffic lights is clearly a numeric variable. The cannonical zero of
  the variable is zero broken traffic lights.
\item
  The attribute \(x_7\) is a ratio variable for much the same reasons as
  for \(x_6\).
\item
  The congestion level is ordinal as it is a discrete variable that can
  be ordered.
\end{itemize}

The statements A, B, and D are all incorrect as they mistake \(x_1\) for something other than an ordinal variable.

\subsection{Question 2}

\begin{enumerate}[label=\Alph*.]
	\item This statement is correct since \(|26 - 19| = 7\) and that is the maximal difference among the respective coordinates of the two vectors.

	\item The metric is
	\[
		d_3(x_{14}, x_{18}) = \sqrt[3]{|26 - 19|^3 + |2 - 0|^3} \approx 7.05.
	\]
	Thus, the statement is incorrect.

	\item The metric is
	\[
		d_1(x_{14}, x_{18}) = |26 - 19| + |2 - 0| = 9.
	\]
	Thus, the statement is incorrect.

	\item The metric is
	\[
		d_4(x_{14}, x_{18}) = \sqrt[4]{|26 - 19|^4 + |2 - 0|^4} \approx 7.01.
	\]
	Thus, the statement is incorrect.

\end{enumerate}

\subsection{Question 3}

\begin{enumerate}[label=\Alph*.]
	\item The explained variance is
	\[
		\frac{13.9^2 + 12.47^2 + 11.48^2 + 10.03^2}{13.9^2 + 12.47^2 + 11.48^2 + 10.03^2 + 9.45^2} \approx 0.87.
	\]
	The statement is therefore correct.

	\item The explained variance is
	\[
		\frac{11.48^2 + 10.03^2 + 9.45^2}{13.9^2 + 12.47^2 + 11.48^2 + 10.03^2 + 9.45^2} \approx 0.48.
	\]
	Thus, the statement is incorrect.

	\item The explained variance is
	\[
		\frac{13.9^2 + 12.47^2}{13.9^2 + 12.47^2 + 11.48^2 + 10.03^2 + 9.45^2} \approx 0.52.
	\]
	Thus, the statement is incorrect.

	\item The explained variance is
	\[
		\frac{13.9^2 + 12.47^2 + 11.48^2 + 10.03^2 + 9.45^2}{13.9^2 + 12.47^2 + 11.48^2 + 10.03^2 + 9.45^2} \approx 0.72.
	\]
	Thus, the statement is incorrect.

\end{enumerate}

\subsection{Question 4}

\begin{enumerate}[label=\Alph*.]
	\item Such an observation will typically have a negative value as the high values of the third and fourth coordinates will make the negative coordinates of the principal component dominate. The statement is false.

	\item For much the same reasons as in A, such an observation will typically have a negative value. This statement is false.

	\item For such an observation, the positive value of the second coordinate of the principal component will dominate the sum in the dot product. The value will therefore typically be positive. The statement is false.

	\item The only negative coordinate of the principal component is the first one. As the observation has a low value in this position, and high values everywhere else, the dot product will typically be positive. The statement is correct.
	\end{enumerate}

\subsection{Question 5}

We are given the following data:
\[
	n = 20000, M_{11} = 2, M_{01} = 5, M_{10} = 6.
\]
The Jaccard similarity of the documents is
\[
	\frac{M_{11}}{M_{11} + M_{01} + M_{10}} \approx 0.1538.
\]
The correct answer is A.

\subsection{Question 6}

The probability \(p(\hat{x}_2 = 0 \, | \, y = 2)\) can be found by marginalizing on \(\hat{x}_7\). This results in the probability
\[
	p(\hat{x}_2 = 0 \, | \, y = 2) = p(\hat{x}_2 = 0, \hat{x}_7 = 0 \, | \, y = 2)
	+ p(\hat{x}_2 = 0, \hat{x}_7 = 1 \, | \, y = 2) = 0.81 + 0.03 = 0.84.
\]
The correct answer is B.

\end{document}
